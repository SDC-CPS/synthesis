\section{Introduction}
\label{sec:intro}

The manufacturing industry represents 12\% of the US GDP
\cite{mfg-innov-report}. One of the key performance indicators for a
manufacturing system is the Overall Equipment Effectiveness (OEE),  capturing
the availability, performance and quality of the production system.  Worldwide
studies indicate that the average OEE in manufacturing plants is 60\%, whereas
world-class OEE is considered to be greater than 85\% \cite{Ahmad2002171}.  One
of the important contributors to this low OEE is unscheduled downtime, caused
by random faults, machine degradation, and increasingly, cyberattacks
\cite{ics-cert,mfg-cyber-14,zetter15,hon2005performance,jonsson1999evaluation,ahmad2002establishing}.

One way to reduce this downtime and improve the overall effectiveness and
management of such manufacturing systems is to develop a {\em global view} of
the system. Such a view will enable \ca early detection of disruptions, \cb
timely isolation of affected component(s) and even \cc potential
reprogramming of parts of (or even the whole) system. Importantly, this
centralized ability to reprogram the system based on a global view also enables
the system operator to quickly react to changing demand, allowing for new parts
to be introduced in an existing manufacturing system, for improved
profitability. We call our approach {\em software-defined control}
(SDC)\footnote{Inspired by the concepts from software-defined networking
(SDN)~\cite{mckeown2008openflow}.}. The key idea of SDC is to {\em separate the logical (control) plane of
decision making from the management of the physical components and sensory
information} for large-scale discrete manufacturing systems. All the
decision-making logic is combined in a single, centralized {\em control plane},
called the ``cyber-physical control plane'' or {\em controller\/} for short in this paper.
The controller has a global view of
the system and it orchestrates  the routing of both physical parts and scheduling of operations in machines by providing detailed instructions to cells in the factory floor. 
Thus, SDC has several similarities with SDNs. In contrast with 
SDNs though, the physical components (akin to packets being routed) have 
different properties -- \ci the queue sizes (buffers) are limited, 
\cii we cannot just ``drop'' widgets like we do with SDN packets and
\ciii the time scales are much different -- since manufacturing systems
operate in the millisecond to second ranges. Thus, we cannot  
directly model/translate SDN concepts to SDC.

Towards this vision, we present \mfname---a formal modeling framework that captures different state components of manufacturing systems, demarcated into {\em plants} and {\em
controllers}. Plants describe the floor plan of the factory -- essentially the
various physical components that operate on the parts being manufactured (that
we call ``widgets'' without loss of generalization). Controllers are cyber
components that orchestrate the operations of the plant(s) and the movement of
widgets\footnote{Section \ref{sec:model} provides more details about
each of these.}. While there exists significant work in the area of modeling
manufacturing systems (\eg \cite{desrochers1995applications,askin1993modeling,cassandras1993discrete,zhou1999modeling,jain2001virtual,ezpeleta1995petri}),
none of them capture the cyber and physical aspects of the system. Modern
manufacturing systems have significant complexity due to the interplay between
the software and hardware components.  \mfname captures these aspects at a level of abstraction that makes it both expressive and useful for simulation and analysis.

\mfname has the power to express a wide variety of {\em discrete}
manufacturing systems. Once 
developed, a \mfname models can be used to reap many benefits of rigorous models, \eg~
\ca computer-aided  analysis and verification,
\cb security analysis (detection of threats, attacks, {\em etc.} and also developing countermeasures),
\cc ensuring that real-time timing requirements are met, 
\cd synthesis of controller code for some (or all parts) of the plant and
\ce monitoring widgets through given plant layouts.
This is akin to the results in the SDN field where the controller (with
its global view of the system) can enable verification (\eg \cite{sdn:rei2012,khurshid2012veriflow}), consistency checks\cite{liu2015,liu2015-2},
dependency checking~\cite{mahajan2013consistent}, synthesizing network
updates~\cite{NoyesWCF13}, system updates~\cite{katta2013incremental,ghorbani2012walk}
and security~\cite{sdn:scott-hayward2016} to name just a few. We believe that rigorously developed control architectures and strategies can significantly reduce unscheduled downtime.

We use both a realistic testbed and a synthetic model for evaluation and analysis (Section~\ref{sec:case_study}).
We are able to record the throughput, end-to-end time and load for varying product manufacturing requirements and plant topologies.

Hence, the high-level contributions of this paper are:
\begin{enumerate}
\item \mfname, a \emph{formal modeling framework} that can capture the cyber and physical
properties of discrete manufacturing systems. [Sections \ref{sec:model}]
\item An illustration of the expressive power of the framework. [Section \ref{subsec:smart_setup}]
\item Implementation of a baseline controller to illustrate the capabilities of \mfname. [Section \ref{sec:baseline-controller}]
\item An open source simulator for \mfname models. [Section \ref{sec:simulation}]
\end{enumerate}

We first start with some background material and a description of the system model.
